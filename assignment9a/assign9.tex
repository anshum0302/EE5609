\documentclass[journal,12pt,twocolumn]{IEEEtran}
%
\usepackage{setspace}
\usepackage{gensymb}
%\doublespacing
\singlespacing
%\usepackage{graphicx}
%\usepackage{amssymb}
%\usepackage{relsize}
\usepackage[cmex10]{amsmath}
%\usepackage{amsthm}
%\interdisplaylinepenalty=2500
%\savesymbol{iint}
%\usepackage{txfonts}
%\restoresymbol{TXF}{iint}
%\usepackage{wasysym}
\usepackage{amsthm}
%\usepackage{iithtlc}
\usepackage{mathrsfs}
\usepackage{txfonts}
\usepackage{stfloats}
\usepackage{bm}
\usepackage{cite}
\usepackage{cases}
\usepackage{subfig}
%\usepackage{xtab}
\usepackage{longtable}
\usepackage{multirow}
%\usepackage{algorithm}
%\usepackage{algpseudocode}
\usepackage{enumitem}
\usepackage{mathtools}
\usepackage{steinmetz}
\usepackage{tikz}
\usepackage{circuitikz}
\usepackage{verbatim}
\usepackage{tfrupee}
\usepackage[breaklinks=true]{hyperref}
%\usepackage{stmaryrd}
\usepackage{tkz-euclide} % loads  TikZ and tkz-base
%\usetkzobj{all}
\usetikzlibrary{calc,math,backgrounds}
\usepackage{caption}
\usepackage{listings}
    \usepackage{color}                          %%
    \usepackage{array}                          %%
    \usepackage{longtable}                      %%
    \usepackage{calc}                           %%
    \usepackage{multirow}                       %%
    \usepackage{hhline}                         %%
    \usepackage{ifthen}                         %%
  %optionally (for landscape tables embedded in another document): %%
    \usepackage{lscape}     
\usepackage{multicol}
\usepackage{chngcntr}
%\usepackage{enumerate}
%\usepackage{wasysym}
%\newcounter{MYtempeqncnt}
\DeclareMathOperator*{\Res}{Res}
%\renewcommand{\baselinestretch}{2}
\renewcommand\thesection{\arabic{section}}
\renewcommand\thesubsection{\thesection.\arabic{subsection}}
\renewcommand\thesubsubsection{\thesubsection.\arabic{subsubsection}}
\renewcommand\thesectiondis{\arabic{section}}
\renewcommand\thesubsectiondis{\thesectiondis.\arabic{subsection}}
\renewcommand\thesubsubsectiondis{\thesubsectiondis.\arabic{subsubsection}}
% correct bad hyphenation here
\hyphenation{op-tical net-works semi-conduc-tor}
\def\inputGnumericTable{}                                 %%
\lstset{
%language=C,
frame=single, 
breaklines=true,
columns=fullflexible
}
%\lstset{
%language=tex,
%frame=single, 
%breaklines=true
%}
\begin{document}
%
\newtheorem{theorem}{Theorem}[section]
\newtheorem{problem}{Problem}
\newtheorem{proposition}{Proposition}[section]
\newtheorem{lemma}{Lemma}[section]
\newtheorem{corollary}[theorem]{Corollary}
\newtheorem{example}{Example}[section]
\newtheorem{definition}[problem]{Definition}
%\newtheorem{thm}{Theorem}[section] 
%\newtheorem{defn}[thm]{Definition}
%\newtheorem{algorithm}{Algorithm}[section]
%\newtheorem{cor}{Corollary}
\newcommand{\BEQA}{\begin{eqnarray}}
\newcommand{\EEQA}{\end{eqnarray}}
\newcommand{\define}{\stackrel{\triangle}{=}}
\bibliographystyle{IEEEtran}
%\bibliographystyle{ieeetr}
\providecommand{\mbf}{\mathbf}
\providecommand{\pr}[1]{\ensuremath{\Pr\left(#1\right)}}
\providecommand{\qfunc}[1]{\ensuremath{Q\left(#1\right)}}
\providecommand{\sbrak}[1]{\ensuremath{{}\left[#1\right]}}
\providecommand{\lsbrak}[1]{\ensuremath{{}\left[#1\right.}}
\providecommand{\rsbrak}[1]{\ensuremath{{}\left.#1\right]}}
\providecommand{\brak}[1]{\ensuremath{\left(#1\right)}}
\providecommand{\lbrak}[1]{\ensuremath{\left(#1\right.}}
\providecommand{\rbrak}[1]{\ensuremath{\left.#1\right)}}
\providecommand{\cbrak}[1]{\ensuremath{\left\{#1\right\}}}
\providecommand{\lcbrak}[1]{\ensuremath{\left\{#1\right.}}
\providecommand{\rcbrak}[1]{\ensuremath{\left.#1\right\}}}
\theoremstyle{remark}
\newtheorem{rem}{Remark}
\newcommand{\sgn}{\mathop{\mathrm{sgn}}}
\providecommand{\abs}[1]{\left\vert#1\right\vert}
\providecommand{\res}[1]{\Res\displaylimits_{#1}} 
\providecommand{\norm}[1]{\left\lVert#1\right\rVert}
%\providecommand{\norm}[1]{\lVert#1\rVert}
\providecommand{\mtx}[1]{\mathbf{#1}}
\providecommand{\mean}[1]{E\left[ #1 \right]}
\providecommand{\fourier}{\overset{\mathcal{F}}{ \rightleftharpoons}}
%\providecommand{\hilbert}{\overset{\mathcal{H}}{ \rightleftharpoons}}
\providecommand{\system}{\overset{\mathcal{H}}{ \longleftrightarrow}}
	%\newcommand{\solution}[2]{\textbf{Solution:}{#1}}
\newcommand{\solution}{\noindent \textbf{Solution: }}
\newcommand{\cosec}{\,\text{cosec}\,}
\providecommand{\dec}[2]{\ensuremath{\overset{#1}{\underset{#2}{\gtrless}}}}
\newcommand{\myvec}[1]{\ensuremath{\begin{pmatrix}#1\end{pmatrix}}}
\newcommand{\mydet}[1]{\ensuremath{\begin{vmatrix}#1\end{vmatrix}}}
%\numberwithin{equation}{section}
\numberwithin{equation}{subsection}
%\numberwithin{problem}{section}
%\numberwithin{definition}{section}
\makeatletter
\@addtoreset{figure}{problem}
\makeatother
\let\StandardTheFigure\thefigure
\let\vec\mathbf
%\renewcommand{\thefigure}{\theproblem.\arabic{figure}}
\renewcommand{\thefigure}{\theproblem}
%\setlist[enumerate,1]{before=\renewcommand\theequation{\theenumi.\arabic{equation}}
%\counterwithin{equation}{enumi}
%\renewcommand{\theequation}{\arabic{subsection}.\arabic{equation}}
\def\putbox#1#2#3{\makebox[0in][l]{\makebox[#1][l]{}\raisebox{\baselineskip}[0in][0in]{\raisebox{#2}[0in][0in]{#3}}}}
     \def\rightbox#1{\makebox[0in][r]{#1}}
     \def\centbox#1{\makebox[0in]{#1}}
     \def\topbox#1{\raisebox{-\baselineskip}[0in][0in]{#1}}
     \def\midbox#1{\raisebox{-0.5\baselineskip}[0in][0in]{#1}}
\vspace{3cm}
\title{Matrix Theory(EE5609) Assignment 9}
\author{Anshum Agrawal \\ Roll No- AI20MTECH11006}
%
\maketitle
\newpage
%\tableofcontents
\bigskip
\renewcommand{\thefigure}{\theenumi}
\renewcommand{\thetable}{\arabic{table}}
%\renewcommand{\thetable}{\theenumi}
%\renewcommand{\theequation}{\theenumi}
%\begin{abstract}
%%\boldmath
\begin{abstract}
Given $\vec{A}\vec{B}$ = -$\vec{B}\vec{A}$, this documents finds value of trace($\vec{A}$) and trace($\vec{B}$).   
\end{abstract}
%
Download latex-tikz codes from 
%
\begin{lstlisting}
https://github.com/anshum0302/EE5609/blob/master/assignment9/assign9.tex
\end{lstlisting}
%
\section{\textbf{PROBLEM STATEMENT}}
Let $\vec{A}$ and $\vec{B}$ be real invertible matrices such that 
\begin{align}
    \vec{AB}=-\vec{BA}\label{eq1}.
\end{align}
Then
\begin{enumerate}
    \item trace{$\vec{A}$} = trace($\vec{B}$) = 0
    \item trace{$\vec{A}$} = trace($\vec{B}$) = 1
    \item trace{$\vec{A}$} = 0, trace($\vec{B}$) = 1
    \item trace($\vec{A}$) = 1, trace($\vec{B}$) = 0
\end{enumerate}
\subsection*{\textbf{Solution:}}
\begin{table}[h!]
\begin{center}
\begin{tabular}{|m{2cm}|m{6cm}|}\hline
        Definition & Matrix $\vec{A}$ is said to be similar to matrix $\vec{B}$ if there exists matrix $\vec{P}$\\& such that $\vec{A}$ = $\vec{P}\vec{B}\vec{P}^{-1}$\\
        \hline
        Properties & Similar matrices have same eigenvalues\\&
        Sum of eigenvalue of a matrix equals its trace\\
        \hline
\end{tabular}
\end{center}
\caption{Similar matrices and Properties}
\label{tab:my_label}
\end{table}
\begin{table}[h!]
\begin{center}
\begin{tabular}{|m{2.2cm}|m{6.3cm}|}\hline
        trace($\vec{A}$) = 0 trace($\vec{B}$) = 0 & From $\eqref{eq1}$ we have $\vec{AB}$=-$\vec{BA}$ or $\vec{A}$ = $\vec{B}$(-$\vec{A}$)$\vec{B}^{-1}$. So, matrix $\vec{A}$ and (-$\vec{A}$) are similar.From properties of similar matrices their eigenvalues are same.Also sum of eigenvalue = trace of matrix.Therefore trace($\vec{A}$) = trace(-$\vec{A}$) or trace($\vec{A}$) = -trace($\vec{A}$) or trace($\vec{A}$) = 0. Similarly we can write $\vec{B}$ = $\vec{A}$(-$\vec{B}$)$\vec{A}^{-1}$.So, matrix $\vec{B}$ and (-$\vec{B}$) are similar.Therefore similar to trace($\vec{A}$), trace($\vec{B}$) = 0. So this statement is true \\
        \hline
        trace($\vec{A}$) = 1 trace($\vec{B}$) = 1 & From $\eqref{eq1}$ we have $\vec{AB}$=-$\vec{BA}$ or $\vec{A}$ = $\vec{B}$(-$\vec{A}$)$\vec{B}^{-1}$. So, matrix $\vec{A}$ and (-$\vec{A}$) are similar.From properties of similar matrices their eigenvalues are same.Also sum of eigenvalue = trace of matrix.Therefore trace($\vec{A}$) = trace(-$\vec{A}$) or trace($\vec{A}$) = -trace($\vec{A}$) or trace($\vec{A}$) = 0. As trace($\vec{A}$) = 0 this statement is false\\
        \hline
        trace($\vec{A}$) = 0 trace($\vec{B}$) = 1 & From $\eqref{eq1}$ we have $\vec{AB}$=-$\vec{BA}$ or $\vec{B}$ = $\vec{A}$(-$\vec{B}$)$\vec{A}^{-1}$. So, matrix $\vec{B}$ and (-$\vec{B}$) are similar.From properties of similar matrices their eigenvalues are same.Also sum of eigenvalue = trace of matrix.Therefore trace($\vec{B}$) = trace(-$\vec{B}$) or trace($\vec{B}$) = -trace($\vec{B}$) or trace($\vec{B}$) = 0. As trace($\vec{B}$) = 0 this statement is false\\
        \hline
        trace($\vec{A}$) = 1 trace($\vec{B}$) = 0 & From $\eqref{eq1}$ we have $\vec{AB}$=-$\vec{BA}$ or $\vec{A}$ = $\vec{B}$(-$\vec{A}$)$\vec{B}^{-1}$. So, matrix $\vec{A}$ and (-$\vec{A}$) are similar.From properties of similar matrices their eigenvalues are same.Also sum of eigenvalue = trace of matrix.Therefore trace($\vec{A}$) = trace(-$\vec{A}$) or trace($\vec{A}$) = -trace($\vec{A}$) or trace($\vec{A}$) = 0. As trace($\vec{A}$) = 0 this statement is false\\
        \hline
        
    \end{tabular}
    \end{center}
    \caption{Calculation of trace}
    \label{tab:my_label}
\end{table}
\end{document}
