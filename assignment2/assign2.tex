\documentclass[journal,12pt,twocolumn]{IEEEtran}
%
\usepackage{setspace}
\usepackage{gensymb}
%\doublespacing
\singlespacing

%\usepackage{graphicx}
%\usepackage{amssymb}
%\usepackage{relsize}
\usepackage[cmex10]{amsmath}
%\usepackage{amsthm}
%\interdisplaylinepenalty=2500
%\savesymbol{iint}
%\usepackage{txfonts}
%\restoresymbol{TXF}{iint}
%\usepackage{wasysym}
\usepackage{amsthm}
%\usepackage{iithtlc}
\usepackage{mathrsfs}
\usepackage{txfonts}
\usepackage{stfloats}
\usepackage{bm}
\usepackage{cite}
\usepackage{cases}
\usepackage{subfig}
%\usepackage{xtab}
\usepackage{longtable}
\usepackage{multirow}
%\usepackage{algorithm}
%\usepackage{algpseudocode}
\usepackage{enumitem}
\usepackage{mathtools}
\usepackage{steinmetz}
\usepackage{tikz}
\usepackage{circuitikz}
\usepackage{verbatim}
\usepackage{tfrupee}
\usepackage[breaklinks=true]{hyperref}
%\usepackage{stmaryrd}
\usepackage{tkz-euclide} % loads  TikZ and tkz-base
%\usetkzobj{all}
\usetikzlibrary{calc,math}
\usepackage{listings}
    \usepackage{color}                          %%
    \usepackage{array}                          %%
    \usepackage{longtable}                      %%
    \usepackage{calc}                           %%
    \usepackage{multirow}                       %%
    \usepackage{hhline}                         %%
    \usepackage{ifthen}                         %%
  %optionally (for landscape tables embedded in another document): %%
    \usepackage{lscape}     
\usepackage{multicol}
\usepackage{chngcntr}
%\usepackage{enumerate}
%\usepackage{wasysym}
%\newcounter{MYtempeqncnt}
\DeclareMathOperator*{\Res}{Res}
%\renewcommand{\baselinestretch}{2}
\renewcommand\thesection{\arabic{section}}
\renewcommand\thesubsection{\thesection.\arabic{subsection}}
\renewcommand\thesubsubsection{\thesubsection.\arabic{subsubsection}}
\renewcommand\thesectiondis{\arabic{section}}
\renewcommand\thesubsectiondis{\thesectiondis.\arabic{subsection}}
\renewcommand\thesubsubsectiondis{\thesubsectiondis.\arabic{subsubsection}}
% correct bad hyphenation here
\hyphenation{op-tical net-works semi-conduc-tor}
\def\inputGnumericTable{}                                 %%
\lstset{
%language=C,
frame=single, 
breaklines=true,
columns=fullflexible
}
%\lstset{
%language=tex,
%frame=single, 
%breaklines=true
%}
\begin{document}
%
\newtheorem{theorem}{Theorem}[section]
\newtheorem{problem}{Problem}
\newtheorem{proposition}{Proposition}[section]
\newtheorem{lemma}{Lemma}[section]
\newtheorem{corollary}[theorem]{Corollary}
\newtheorem{example}{Example}[section]
\newtheorem{definition}[problem]{Definition}
%\newtheorem{thm}{Theorem}[section] 
%\newtheorem{defn}[thm]{Definition}
%\newtheorem{algorithm}{Algorithm}[section]
%\newtheorem{cor}{Corollary}
\newcommand{\BEQA}{\begin{eqnarray}}
\newcommand{\EEQA}{\end{eqnarray}}
\newcommand{\define}{\stackrel{\triangle}{=}}
\bibliographystyle{IEEEtran}
%\bibliographystyle{ieeetr}
\providecommand{\mbf}{\mathbf}
\providecommand{\pr}[1]{\ensuremath{\Pr\left(#1\right)}}
\providecommand{\qfunc}[1]{\ensuremath{Q\left(#1\right)}}
\providecommand{\sbrak}[1]{\ensuremath{{}\left[#1\right]}}
\providecommand{\lsbrak}[1]{\ensuremath{{}\left[#1\right.}}
\providecommand{\rsbrak}[1]{\ensuremath{{}\left.#1\right]}}
\providecommand{\brak}[1]{\ensuremath{\left(#1\right)}}
\providecommand{\lbrak}[1]{\ensuremath{\left(#1\right.}}
\providecommand{\rbrak}[1]{\ensuremath{\left.#1\right)}}
\providecommand{\cbrak}[1]{\ensuremath{\left\{#1\right\}}}
\providecommand{\lcbrak}[1]{\ensuremath{\left\{#1\right.}}
\providecommand{\rcbrak}[1]{\ensuremath{\left.#1\right\}}}
\theoremstyle{remark}
\newtheorem{rem}{Remark}
\newcommand{\sgn}{\mathop{\mathrm{sgn}}}
\providecommand{\abs}[1]{\left\vert#1\right\vert}
\providecommand{\res}[1]{\Res\displaylimits_{#1}} 
\providecommand{\norm}[1]{\left\lVert#1\right\rVert}
%\providecommand{\norm}[1]{\lVert#1\rVert}
\providecommand{\mtx}[1]{\mathbf{#1}}
\providecommand{\mean}[1]{E\left[ #1 \right]}
\providecommand{\fourier}{\overset{\mathcal{F}}{ \rightleftharpoons}}
%\providecommand{\hilbert}{\overset{\mathcal{H}}{ \rightleftharpoons}}
\providecommand{\system}{\overset{\mathcal{H}}{ \longleftrightarrow}}
	%\newcommand{\solution}[2]{\textbf{Solution:}{#1}}
\newcommand{\solution}{\noindent \textbf{Solution: }}
\newcommand{\cosec}{\,\text{cosec}\,}
\providecommand{\dec}[2]{\ensuremath{\overset{#1}{\underset{#2}{\gtrless}}}}
\newcommand{\myvec}[1]{\ensuremath{\begin{pmatrix}#1\end{pmatrix}}}
\newcommand{\mydet}[1]{\ensuremath{\begin{vmatrix}#1\end{vmatrix}}}
\newcommand{\mymat}[1]{\ensuremath{\begin{bmatrix}#1\end{bmatrix}}}
%\numberwithin{equation}{section}
\numberwithin{equation}{subsection}
%\numberwithin{problem}{section}
%\numberwithin{definition}{section}
\makeatletter
\@addtoreset{figure}{problem}
\makeatother
\let\StandardTheFigure\thefigure
\let\vec\mathbf
%\renewcommand{\thefigure}{\theproblem.\arabic{figure}}
\renewcommand{\thefigure}{\theproblem}
%\setlist[enumerate,1]{before=\renewcommand\theequation{\theenumi.\arabic{equation}}
%\counterwithin{equation}{enumi}
%\renewcommand{\theequation}{\arabic{subsection}.\arabic{equation}}
\def\putbox#1#2#3{\makebox[0in][l]{\makebox[#1][l]{}\raisebox{\baselineskip}[0in][0in]{\raisebox{#2}[0in][0in]{#3}}}}
     \def\rightbox#1{\makebox[0in][r]{#1}}
     \def\centbox#1{\makebox[0in]{#1}}
     \def\topbox#1{\raisebox{-\baselineskip}[0in][0in]{#1}}
     \def\midbox#1{\raisebox{-0.5\baselineskip}[0in][0in]{#1}}
\vspace{3cm}
\title{Matrix Theory(EE5609) Assignment 2}
\author{Anshum Agrawal \\ Roll No- AI20MTECH11006}
%
\maketitle
\newpage
%\tableofcontents
\bigskip
\renewcommand{\thefigure}{\theenumi}
\renewcommand{\thetable}{\theenumi}
%\renewcommand{\theequation}{\theenumi}
%\begin{abstract}
%%\boldmath
\begin{abstract}
  This Assignment finds investment to be made in two different bonds to get the desired interest.
\end{abstract}
Download all python codes from 
%
\begin{lstlisting}
https://github.com/anshum0302/EE5609/blob/master/assignment2/solu2.py
\end{lstlisting}
%
and latex-tikz codes from 
%
\begin{lstlisting}
https://github.com/anshum0302/EE5609/blob/master/assignment2/assign2.tex
\end{lstlisting}
%
\section{\textbf{PROBLEM STATEMENT}}
A trust fund has \rupee{30000} that must be invested in two different types of bonds.The first bond pays 5\% interest per year, and the second bond pays 7\% interest per year.Using matrix multiplication, determine how to divide \rupee{30000} among the two types of bonds.If the trust fund must obtain an annual total interest of : a) \rupee{1800} b) \rupee{2000}.

\section{\textbf{Solution}}
Let \rupee{30000} be divided into two part $xa1$ and $xa2$ in part $\vec{a)}$,and into two part $xb1$ and $xb2$ in part $\vec{b)}$. Then $xa1,xa2,xb1,xb2$ satisfies following equations
\begin{align}
  xa1+xa2 &= 30000\label{eq1}\\
  0.05xa1+0.07xa2 &= 1800\label{eq2}\\
  xb1+xb2 &= 30000\label{eq3}\\
  0.05xb1+0.07xb2 &= 2000\label{eq4}
\end{align}
From \eqref{eq1} and \eqref{eq2} we get
\begin{align}
  \mymat{1&1\\0.05&0.07}\mymat{xa1\\xa2} = \mymat{30000\\1800}
\end{align}
and from \eqref{eq3} and \eqref{eq4} we get
\begin{align}
  \mymat{1&1\\0.05&0.07}\mymat{xb1\\xb2} = \mymat{30000\\2000}
\end{align}
Combining the two we get
\begin{align}
  &\mymat{1&1\\0.05&0.07}\mymat{xa1&xb1\\xa2&xb2} = \mymat{30000&30000\\1800&2000} \nonumber \\
  &\xleftrightarrow{R_2 = R_2 - 0.05 R_1}\mymat{1&1\\0&0.02}\mymat{xa1&xb1\\xa2&xb2} = \mymat{30000&30000\\300&500} \nonumber \\
  &\xleftrightarrow{R_2 = 50R_2}\mymat{1&1\\0&1}\mymat{xa1&xb1\\xa2&xb2} = \mymat{30000&30000\\15000&25000} \nonumber \\
  &\xleftrightarrow{R_1 = R_1-R_2}\mymat{1&0\\0&1}\mymat{xa1&xb1\\xa2&xb2} = \mymat{15000&5000\\15000&25000}\label{ans}
\end{align}
From \eqref{ans} we get $xa1$=\rupee{15000}, $xa2$=\rupee{15000}, $xb1$=\rupee{5000} and $xb2$=\rupee{25000}.Therefore to get an annual total interest of \rupee{1800} trust must invest \rupee{15000} in first bond and \rupee{15000} in second bond and to get an annual interest of \rupee{2000} trust must invest \rupee{5000} in first bond and \rupee{25000} in second bond.
\end{document}
