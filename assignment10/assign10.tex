\documentclass[journal,12pt,twocolumn]{IEEEtran}
%
\usepackage{setspace}
\usepackage{gensymb}
%\doublespacing
\singlespacing
%\usepackage{graphicx}
%\usepackage{amssymb}
%\usepackage{relsize}
\usepackage[cmex10]{amsmath}
%\usepackage{amsthm}
%\interdisplaylinepenalty=2500
%\savesymbol{iint}
%\usepackage{txfonts}
%\restoresymbol{TXF}{iint}
%\usepackage{wasysym}
\usepackage{amsthm}
%\usepackage{iithtlc}
\usepackage{mathrsfs}
\usepackage{txfonts}
\usepackage{stfloats}
\usepackage{bm}
\usepackage{cite}
\usepackage{cases}
\usepackage{subfig}
%\usepackage{xtab}
\usepackage{longtable}
\usepackage{multirow}
%\usepackage{algorithm}
%\usepackage{algpseudocode}
\usepackage{enumitem}
\usepackage{mathtools}
\usepackage{steinmetz}
\usepackage{tikz}
\usepackage{circuitikz}
\usepackage{verbatim}
\usepackage{tfrupee}
\usepackage[breaklinks=true]{hyperref}
%\usepackage{stmaryrd}
\usepackage{tkz-euclide} % loads  TikZ and tkz-base
%\usetkzobj{all}
\usetikzlibrary{calc,math,backgrounds}
\usepackage{caption}
\usepackage{listings}
    \usepackage{color}                          %%
    \usepackage{array}                          %%
    \usepackage{longtable}                      %%
    \usepackage{calc}                           %%
    \usepackage{multirow}                       %%
    \usepackage{hhline}                         %%
    \usepackage{ifthen}                         %%
  %optionally (for landscape tables embedded in another document): %%
    \usepackage{lscape}     
\usepackage{multicol}
\usepackage{chngcntr}
%\usepackage{enumerate}
%\usepackage{wasysym}
%\newcounter{MYtempeqncnt}
\DeclareMathOperator*{\Res}{Res}
%\renewcommand{\baselinestretch}{2}
\renewcommand\thesection{\arabic{section}}
\renewcommand\thesubsection{\thesection.\arabic{subsection}}
\renewcommand\thesubsubsection{\thesubsection.\arabic{subsubsection}}
\renewcommand\thesectiondis{\arabic{section}}
\renewcommand\thesubsectiondis{\thesectiondis.\arabic{subsection}}
\renewcommand\thesubsubsectiondis{\thesubsectiondis.\arabic{subsubsection}}
% correct bad hyphenation here
\hyphenation{op-tical net-works semi-conduc-tor}
\def\inputGnumericTable{}                                 %%
\lstset{
%language=C,
frame=single, 
breaklines=true,
columns=fullflexible
}
%\lstset{
%language=tex,
%frame=single, 
%breaklines=true
%}
\begin{document}
%
\newtheorem{theorem}{Theorem}[section]
\newtheorem{problem}{Problem}
\newtheorem{proposition}{Proposition}[section]
\newtheorem{lemma}{Lemma}[section]
\newtheorem{corollary}[theorem]{Corollary}
\newtheorem{example}{Example}[section]
\newtheorem{definition}[problem]{Definition}
%\newtheorem{thm}{Theorem}[section] 
%\newtheorem{defn}[thm]{Definition}
%\newtheorem{algorithm}{Algorithm}[section]
%\newtheorem{cor}{Corollary}
\newcommand{\BEQA}{\begin{eqnarray}}
\newcommand{\EEQA}{\end{eqnarray}}
\newcommand{\define}{\stackrel{\triangle}{=}}
\bibliographystyle{IEEEtran}
%\bibliographystyle{ieeetr}
\providecommand{\mbf}{\mathbf}
\providecommand{\pr}[1]{\ensuremath{\Pr\left(#1\right)}}
\providecommand{\qfunc}[1]{\ensuremath{Q\left(#1\right)}}
\providecommand{\sbrak}[1]{\ensuremath{{}\left[#1\right]}}
\providecommand{\lsbrak}[1]{\ensuremath{{}\left[#1\right.}}
\providecommand{\rsbrak}[1]{\ensuremath{{}\left.#1\right]}}
\providecommand{\brak}[1]{\ensuremath{\left(#1\right)}}
\providecommand{\lbrak}[1]{\ensuremath{\left(#1\right.}}
\providecommand{\rbrak}[1]{\ensuremath{\left.#1\right)}}
\providecommand{\cbrak}[1]{\ensuremath{\left\{#1\right\}}}
\providecommand{\lcbrak}[1]{\ensuremath{\left\{#1\right.}}
\providecommand{\rcbrak}[1]{\ensuremath{\left.#1\right\}}}
\theoremstyle{remark}
\newtheorem{rem}{Remark}
\newcommand{\sgn}{\mathop{\mathrm{sgn}}}
\providecommand{\abs}[1]{\left\vert#1\right\vert}
\providecommand{\res}[1]{\Res\displaylimits_{#1}} 
\providecommand{\norm}[1]{\left\lVert#1\right\rVert}
%\providecommand{\norm}[1]{\lVert#1\rVert}
\providecommand{\mtx}[1]{\mathbf{#1}}
\providecommand{\mean}[1]{E\left[ #1 \right]}
\providecommand{\fourier}{\overset{\mathcal{F}}{ \rightleftharpoons}}
%\providecommand{\hilbert}{\overset{\mathcal{H}}{ \rightleftharpoons}}
\providecommand{\system}{\overset{\mathcal{H}}{ \longleftrightarrow}}
	%\newcommand{\solution}[2]{\textbf{Solution:}{#1}}
\newcommand{\solution}{\noindent \textbf{Solution: }}
\newcommand{\cosec}{\,\text{cosec}\,}
\providecommand{\dec}[2]{\ensuremath{\overset{#1}{\underset{#2}{\gtrless}}}}
\newcommand{\myvec}[1]{\ensuremath{\begin{pmatrix}#1\end{pmatrix}}}
\newcommand{\mydet}[1]{\ensuremath{\begin{vmatrix}#1\end{vmatrix}}}
%\numberwithin{equation}{section}
\numberwithin{equation}{subsection}
%\numberwithin{problem}{section}
%\numberwithin{definition}{section}
\makeatletter
\@addtoreset{figure}{problem}
\makeatother
\let\StandardTheFigure\thefigure
\let\vec\mathbf
%\renewcommand{\thefigure}{\theproblem.\arabic{figure}}
\renewcommand{\thefigure}{\theproblem}
%\setlist[enumerate,1]{before=\renewcommand\theequation{\theenumi.\arabic{equation}}
%\counterwithin{equation}{enumi}
%\renewcommand{\theequation}{\arabic{subsection}.\arabic{equation}}
\def\putbox#1#2#3{\makebox[0in][l]{\makebox[#1][l]{}\raisebox{\baselineskip}[0in][0in]{\raisebox{#2}[0in][0in]{#3}}}}
     \def\rightbox#1{\makebox[0in][r]{#1}}
     \def\centbox#1{\makebox[0in]{#1}}
     \def\topbox#1{\raisebox{-\baselineskip}[0in][0in]{#1}}
     \def\midbox#1{\raisebox{-0.5\baselineskip}[0in][0in]{#1}}
\vspace{3cm}
\title{Matrix Theory(EE5609) Assignment 10}
\author{Anshum Agrawal \\ Roll No- AI20MTECH11006}
%
\maketitle
\newpage
%\tableofcontents
\bigskip
\renewcommand{\thefigure}{\theenumi}
\renewcommand{\thetable}{\arabic{table}}
%\renewcommand{\thetable}{\theenumi}
%\renewcommand{\theequation}{\theenumi}
%\begin{abstract}
%%\boldmath
\begin{abstract}
Given matrix $\vec{T}$ find matrix $\vec{T[C]}$ which represents matrix $\vec{T}$ with respect to basis $\vec{C}$.
\end{abstract}
%
Download latex-tikz codes from 
%
\begin{lstlisting}
https://github.com/anshum0302/EE5609/blob/master/assignment10/assign10.tex
\end{lstlisting}
%
\section{\textbf{PROBLEM STATEMENT}}
Let $\vec{C}$ = $\myvec{\myvec{1\\2},\myvec{2\\1}}$ be a basis for $\mathbb{R}^2$ and $\vec{T}$:$\mathbb{R}^2$$\rightarrow$$\mathbb{R}^2$ be defined by $\vec{T}\myvec{x\\y}$ = $\myvec{x+y\\x-2y}$. If $\vec{T[C]}$ represents the matrix of $\vec{T}$ with respect to basis $\vec{C}$ then which of the following is true
\begin{enumerate}
    \item $\vec{T[C]}$ = $\myvec{-3&-2\\3&1}$ 
    \item $\vec{T[C]}$ = $\myvec{3&-2\\-3&1}$
    \item $\vec{T[C]}$ = $\myvec{-3&-1\\3&2}$
    \item $\vec{T[C]}$ = $\myvec{3&-1\\-3&2}$
\end{enumerate}
\subsection*{\textbf{Solution:}}
\begin{table}[h!]
\begin{center}
\begin{tabular}{|m{2cm}|m{6cm}|}\hline
        Linear Transformation and change of Basis & If matrix $\vec{A}$ represents Linear Transformation with respect to standard ordered basis and matrix $\vec{B}$ represents same transformation with respect to basis $\vec{V}$,Then{\begin{align*}
            \vec{B} = \vec{V}^{-1}\vec{A}\vec{V}
        \end{align*}}\\
        \hline
\end{tabular}
\end{center}
\caption{Linear Transformation and change of basis}
\label{tab:my_label}
\end{table}
In above question $\vec{A}$ = $\vec{T}$,$\vec{B}$ = $\vec{T[C]}$,$\vec{V}$ = $\vec{C}$.
\begin{table}[h!]
\begin{center}
\begin{tabular}{|m{2.2cm}|m{6.3cm}|}\hline
        Evaluate $\vec{T}$ & For linear transformation $\vec{T}$ we have {\begin{align*}
            \vec{T}\myvec{x\\y} &= \myvec{x+y\\x-2y}\\
            \vec{T}\myvec{x\\y} &= \myvec{1&1\\1&-2}\myvec{x\\y}\\
            \implies\vec{T} &= \myvec{1&1\\1&-2}
        \end{align*}} \\
        \hline
        Evaluate inverse of basis $\vec{C}$ & To find inverse of matrix $\vec{C}$ we row reduce augmented matrix $\vec{C}$$\vec{I}$ {\begin{align*}
            \myvec{1&2&1&0\\2&1&0&1}\xleftrightarrow[R_2=-\frac{1}{3}R_2]{R_2=R_2-2R_1}\myvec{1&2&1&0\\0&1&\frac{2}{3}&-\frac{1}{3}}\\
            \xleftrightarrow{R_1=R_1-2R_2}\myvec{1&0&-\frac{1}{3}&\frac{2}{3}\\0&1&\frac{2}{3}&-\frac{1}{3}}
        \end{align*}}\\
        &$\therefore$ $\vec{C}^{-1}$ = $\myvec{-\frac{1}{3}&\frac{2}{3}\\\frac{2}{3}&-\frac{1}{3}}$ \\
        \hline
        Evaluate $\vec{TC}$ & {\begin{align*}
            \vec{TC} &= \myvec{1&1\\1&-2}\myvec{1&2\\2&1}\\
            &= \myvec{3&3\\-3&0}
        \end{align*}}\\
        \hline
        Evaluate $\vec{T[C]}$=$\vec{C}^{-1}\vec{TC}$& {\begin{align*}
            \vec{T[C]} &= \vec{C}^{-1}\vec{TC}\\
            &=\myvec{-\frac{1}{3}&\frac{2}{3}\\\frac{2}{3}&-\frac{1}{3}}\myvec{3&3\\-3&0}\\
            \implies \vec{T[C]} &= \myvec{-3&-1\\3&2}
        \end{align*}} \\
        \hline
        Conclusion&Option 3) is correct.Options 1),2) and 4) are incorrect\\
        \hline
    \end{tabular}
    \end{center}
    \caption{Calculation of $\vec{T[C]}$}
    \label{tab:my_label}
\end{table}
\end{document}
